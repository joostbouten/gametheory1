\documentclass[a4paper]{article}
\usepackage{geometry}
\textwidth = 426pt
\usepackage{lmodern}
\usepackage{array}
\usepackage{tabularx}
\usepackage{multirow}
\usepackage{enumerate}
\usepackage{amssymb,amsmath}
\usepackage{ifxetex,ifluatex}
\usepackage{fixltx2e} % provides \textsubscript
\ifnum 0\ifxetex 1\fi\ifluatex 1\fi=0 % if pdftex
  \usepackage[T1]{fontenc}
  \usepackage[utf8]{inputenc}
\else % if luatex or xelatex
  \ifxetex
    \usepackage{mathspec}
  \else
    \usepackage{fontspec}
  \fi
  \defaultfontfeatures{Ligatures=TeX,Scale=MatchLowercase}
\fi
% use upquote if available, for straight quotes in verbatim environments
\IfFileExists{upquote.sty}{\usepackage{upquote}}{}
% use microtype if available
\IfFileExists{microtype.sty}{%
\usepackage[]{microtype}
\UseMicrotypeSet[protrusion]{basicmath} % disable protrusion for tt fonts
}{}
\PassOptionsToPackage{hyphens}{url} % url is loaded by hyperref
\usepackage[unicode=true]{hyperref}
\hypersetup{
            pdfborder={0 0 0},
            breaklinks=true}
\urlstyle{same}  % don't use monospace font for urls
\usepackage{longtable,booktabs}
% Fix footnotes in tables (requires footnote package)
\IfFileExists{footnote.sty}{\usepackage{footnote}\makesavenoteenv{long table}}{}
\IfFileExists{parskip.sty}{%
\usepackage{parskip}
}{% else
\setlength{\parindent}{0pt}
\setlength{\parskip}{6pt plus 2pt minus 1pt}
}
\setlength{\emergencystretch}{3em}  % prevent overfull lines
\providecommand{\tightlist}{%
  \setlength{\itemsep}{0pt}\setlength{\parskip}{0pt}}
\setcounter{secnumdepth}{0}
% Redefines (sub)paragraphs to behave more like sections
\ifx\paragraph\undefined\else
\let\oldparagraph\paragraph
\renewcommand{\paragraph}[1]{\oldparagraph{#1}\mbox{}}
\fi
\ifx\subparagraph\undefined\else
\let\oldsubparagraph\subparagraph
\renewcommand{\subparagraph}[1]{\oldsubparagraph{#1}\mbox{}}
\fi

% set default figure placement to htbp
\makeatletter
\def\fps@figure{htbp}
\makeatother

\author{Group 10: Joost Bouten, Twan Vissers}

\date{25 October, 2017}
\title{Game Theory 2: Problem Set 1}

\begin{document}

\maketitle
%\section{Problem Set 1}\label{problem-set-1}
\subsubsection{Problem 1}
\begin{enumerate}[(a)]
\item
Action spaces:\\
$A_A=[0,1]$\\
$A_T=\{accept,\ reject\}$\\
$A_T=[0,1]$ which represents a cutoff above which to accept the bid $y$.\\
\\
Type spaces:\\
$\Theta_A=E(x)$, i.e. $A$ only has one state of nature.\\
$\Theta_T=x=[0,1]$\\

Strategies:\\
$s_A:\Theta_A\rightarrow A_A$\\
A strategy for $A$ represents an action (bid $y$) given its expectation of firm $T$'s value $x$
\\ 
$s_T: \Theta_T \times A_A \rightarrow A_T$\\ A strategy for $T$ represents an action $accept$ or $reject$ given firm $T$'s value $x$ and given the bid $y$ of $A$, chosen from its action space $A_A$. 
\\ \\
Beliefs:\\
Player $A$'s beliefs about $T$'s types are given by the cumulative probability distribution function $F(x)=x(2-x)$
\\
Player $T$ is certain of the player $A$'s type, it knows exactly what $A$ expects firm $T$'s value to be. 
\\
\\
Payoff functions:\\
$u_A=\begin{cases}
2x-y & \quad \text{if }
s_T=accept\\
0 & \quad \text{if } s_T=reject
\end{cases}
$
\\
$u_T=\begin{cases}
y & \quad \text{if }
s_T=accept\\
x & \quad \text{if } s_T=reject
\end{cases}
$
\\
\item
The expected value of firm $T$ is equal to the expected value of the probability density function of $F(x)$:

$E(x)=\int_0^1x\cdot f(x)dx \text{ with } f(x)=\frac{\partial F(x)}{\partial x}(2x-x^2)=2-2x$\\
$E(x)=\int_0^1\left(2x-2x^2\right)dx=\left[x^2-\frac{2}{3}x^3\right]_0^1=\frac{1}{3}$

Thus, the (unconditional) expected value of firm $T$ equals $\dfrac{1}{3}$.
\\
\item
$A$'s expected value of firm $T$ if firm $T$ accepts the bid is given by:\\
$E(x|x\le y)=\left(\int_0^yx\cdot f(x)dx\right)\big/F(y)=\dfrac{y-\frac{2}{3}y^2}{2-y}$\\

Given this expectation, the expectation of $A$'s utility given that bid $y$ is accepted is:\\
$E(u_A)=E(2x-y|x\le y)=2\cdot E(x|x\le y)-y=2\cdot\dfrac{y-\frac{2}{3}y^2}{2-y}-y$\\
$E(u_A)=\dfrac{2y-\frac{4}{3}y^2}{2-y}-\dfrac{y(2-y)}{2-y}=\dfrac{-\frac{1}{3}y^2}{2-y}$\\
$FOC: \dfrac{\partial E(u_A)}{\partial y}=\dfrac{-\frac{2}{3}y(2-y)-\frac{1}{3}y^2}{(2-y)^2}=0 \iff y=0 \vee y=4$

As can be concluded from the first order criterium, it is optimal for firm $A$ to bid $y=0$ As this will yield the maximum payoff. Furthermore, $y=4$ will result in a mimimum and is thus not a solution of the maximization problem.

In the Bayesian Nash equilibrium of the game, there will be no trade and the equilibrium bid is $y=0$, which will never be accepted by firm $T$. 
\item
The logic behind the equilibrium is called adverse selection because due to a lack of information an undesired result (no trade) will follow from the game, even though trade could benefit both $A$ and $T$.

\end{enumerate}
\subsubsection{Problem 2}
\begin{enumerate}[(a)]
\item
\textsubscript{­}­\(\Theta_{i} = \{ c_{L},\ c_{H}\}\) for
\(i = 1,\ 2\).

Firm \(i\) being of the low cost type \(c_{L}\) occurs with probability
\(v\) while firm \(i\) being of the high cost type \(c_{H}\) occurs with
probability \(1 - v\).


\item \(A_{i} = \left\{ q_{i} \right\}\) for \(i = 1,\ 2\).

\item The strategy for firm\(\text{\ i}\) can be described as follows:
Given the type of firm $i$, firm $i$ chooses an appropriate action for each possible action of the other firm $j$.\\
i.e. \(s_{i}:\ \ \ \ \ \ \Theta_{i} \times A_{j} \rightarrow A_{i}\ \text{, where }i\ne j \)
\item
$\pi_1=\left(a-q_1-q_2-v\cdot c_L- (1-v)\cdot c_H\right)q_1$\\
$\frac{\partial\pi_1}{\partial q_1}=a-2q_1-q_2-v\cdot c_L-(1-v)\cdot c_H=0 \iff q_1=b_1(q_2)=\frac{1}{2}(a-q_2-v\cdot c_L-(1-v)\cdot c_H)$\\ \\
Similarly, $q_2=b_2(q_1)=\frac{1}{2}(a-q_1-v\cdot c_L-(1-v)\cdot c_H)$
\\
We can substitute for $q_2$ in $b_1(q_2)$:
\\
$q_1=\frac{1}{2}(a-\frac{1}{2}(a-q_1-v\cdot c_L-(1-v)\cdot c_H)-v\cdot c_L-(1-v)\cdot c_H)$\\
$\frac{3}{4}q_1=\frac{1}{4}(a-v\cdot c_L-(1-v)\cdot c_H)\iff q_1=\frac{1}{3}(a-v\cdot c_L-(1-v)\cdot c_H)$\\
\\ We can similarly calculate $q_2$.
\\
Thus the symmetric Bayesian Nash equilibrium of this game is:
$$q_i=\frac{1}{3}(a-v\cdot c_L-(1-v)\cdot c_H) \text{ for } i=1,2$$
\end{enumerate}

\subsubsection{Problem 3}
\begin{enumerate}[(a)]
\item
\begin{itemize}
\item If $k_2=k_H$,\\
{\renewcommand{\arraystretch}{2}
\begin{tabular}{cc|c|c|}
      & \multicolumn{1}{c}{} & \multicolumn{2}{c}{Player $2$}\\
      & \multicolumn{1}{c}{} & \multicolumn{1}{c}{$C$}  & \multicolumn{1}{c}{$D$} \\\cline{3-4}
      \multirow{2}*{Player $1$}  & $C$ & $0,0$ & $\underline{0},\underline{R}$ \\\cline{3-4}
      & $D$ & $\underline{R},\underline{0}$ & $\dfrac{R}{2}-k_H,\dfrac{R}{2}-k_H$ \\\cline{3-4}
\end{tabular}
}\\
\\
\item If $k_2=k_L$,\\
{\renewcommand{\arraystretch}{2}
\begin{tabular}{cc|c|c|}
      & \multicolumn{1}{c}{} & \multicolumn{2}{c}{Player $2$}\\
      & \multicolumn{1}{c}{} & \multicolumn{1}{c}{$C$}  & \multicolumn{1}{c}{$D$} \\\cline{3-4}
      \multirow{2}*{Player $1$}  & $C$ & $0,0$ & $\underline{0},\underline{R}$ \\\cline{3-4}
      & $D$ & $\underline{R},0$ & $\dfrac{R}{2}-k_H,\underline{\dfrac{R}{2}-k_L}$ \\\cline{3-4}
\end{tabular}
}
\end{itemize}
\item
\begin{itemize}
\item
If $k_2=k_H$,\\
The game has two pure strategy Nash equilibria:

$$(D, C),(C,D) $$ 

Checking for a mixed strategy equilibrium:

$$u_1(C, \sigma_2)=0$$
$$u_1(D,\sigma_2)=\sigma_{2C}\cdot R+(1-\sigma_{2C})\cdot\left(\dfrac{R}{2}-k_H\right)$$
$$u_1(C, \sigma_2)=u_1(D,\sigma_2) \iff \sigma_{2C}=\dfrac{2k_H-R}{2k_H+R}$$

Similarly for player 2:
$$u_2(\sigma_1, C)=u_2(\sigma_1,D) \iff \sigma_{1C}=\dfrac{2k_H-R}{2k_H+R}$$

Thus there is one mixed strategy nash equilibruim:

$$(\sigma_{1C},\sigma_{1D}),(\sigma_{2C},\sigma_{2D})=\left(\dfrac{2k_H-R}{2k_H+R},\dfrac{2R}{2k_H+R}\right),\left(\dfrac{2k_H-R}{2k_H+R},\dfrac{2R}{2k_H+R}\right)$$

\item If $k_2=k_L$,\\

We can see that for player 2, the strategy $C$ is strictly dominated by $D$. The following payoff matrix remains after deleting strategy $C$ for player 2. 

{\renewcommand{\arraystretch}{2}
\begin{tabular}{cc|c|}
      & \multicolumn{1}{c}{} & \multicolumn{1}{c}{Player $2$}\\
      & \multicolumn{1}{c}{} & \multicolumn{1}{c}{$D$}   \\\cline{3-3}
      \multirow{2}*{Player $1$}  & $C$ & $\underline{0},R$ \\\cline{3-3}
      & $D$  & $\dfrac{R}{2}-k_H,\dfrac{R}{2}-k_L$ \\\cline{3-3}
\end{tabular}
}

We can now see that for player 1 in this matrix, strategy $D$ is strictly dominated by strategy $C$.\\

This means that there is only one unique pure strategy Nash equilibrium:
$$(C,D)$$ 

And there are no mixed strategy Nash equilibria.
\end{itemize}
\item
If player 2 has an airbag, player 2 always prefers driving head on versus swerving to the right as the potential losses from crashing do not outweigh the potential benefits in terms of respect that are generated by driving head on if both players drive head on.  Player 2 will thus always gain from driving head on and player 2 will therefore do so.

If player 2 has an airbag, player 1 will know that player 2 will always drive head on. Knowing this, player one will always gain from swerving to the right, as otherwise an accident will occur of which the costs outweigh the benefits. Player 1 will thus always choose to swerve to the right and have a payoff of 0 instead of a negative payoff.

\item
Type spaces:

$\Theta_1=\{k_H\}$\\
$\Theta_2=\{k_L,k_H\}$\\

Action spaces:

$A_i=\{C,D\}=\{swerve\ right,\ head\ on\} \text{ for }i=1,2$\\


\item
A strategy for player 1 in this game is a function $s_1$ that specifies with which probabilities to drive head on or swerve right given its type $k_H$.

A strategy for player 2 in this game is a function $s_2$ that specifies with which probabilities to drive head on or swerve right for each of its possible types $k_L$ and $k_H$.
\\ \\
That is, $$s_i:\Theta_i\rightarrow A_i \text{ for } i=1,2$$

\item

{\renewcommand{\arraystretch}{2}
\begin{tabularx}{\textwidth}{cc|c|c|c|c|} 
      & \multicolumn{1}{c}{} & \multicolumn{4}{c}{\textbf{2}}\\
      & \multicolumn{1}{c}{} & \multicolumn{1}{c}{$C,C$}  & \multicolumn{1}{c}{$C,D$}& \multicolumn{1}{c}{$D,C$} & \multicolumn{1}{c}{$D,D$}\\
\cline{3-6}
      \multirow{3}*{\textbf{1}}  & $C$ & $0,0$ & $0, (1-p)R$ & $0,p R$ & $\underline{0},\underline{R}$ \\
\cline{3-6}
      & \multirow{2}*{$D$} & $\underline{R},$ & $p R+(1-p)({\dfrac{R}{2}-k_H}) $,  & $p({\dfrac{R}{2}-k_H})+(1-p) R$,  & ${\dfrac{R}{2}-k_H}$, \\
&&$0$	&	$(1-p)({\dfrac{R}{2}-k_H})$ & $\underline{p({\dfrac{R}{2}-k_L})}$ &$p {\dfrac{R}{2}-k_L}+(1-p){\dfrac{R}{2}-k_H}$  \\\cline{3-6}

\end{tabularx}
}\\  \\ \\
By inspection we find the following pure strategy Bayesian Nash equilibrium (PS-BNE) for this game, as underlined in the table above:
$(C, DD)$
\\ \\
However, there could be a situation in which there exists another PS-BNE where player 1 plays D and player 2 plays DC. This can only be the case if the following condition holds:
$$p\cdot\left({\dfrac{R}{2}-k_H}\right)+(1-p)\cdot R \ge 0$$
In case that the condition is satisfied, then $(C,DD)$ and $(D,DC)$ are both perfect strategy Bayesian Nash equilibria. If the condition is not satisfied, only $(C,DD)$ will be a PS-BNE of this Game of Chicken.

Furthermore we can calculate the Bayesian Nash equilibrium in mixed strategies, let $x$ be the probability of player 1 playing $C$, let $y$ be the probability of player 2 playing $C$. If player 2 is of type $k_L$, it will always choose to play $D$, if player 2 is of type $k_H$ it could mix between the two strategies as we have seen in question 3 (b).
\\ \\
The MSE is thus of the form:
$$\left(s_1;s_2(k_L),s_2(k_H)\right)=\left(D \text{ with prob. }x; D, D\text{ with prob. }y\right)$$
Where $y$ satisfies: $$(1-y)R +y\cdot\left(\frac{R}{2}-k_H\right)=0$$
$$\Rightarrow y=\dfrac{2R}{2k_H+R}$$\\ 
And where $x$ satisfies: $$p\left(\frac{R}{2}-k_H\right)+(1-p)\left(x\cdot R+(1-x)\left(\frac{R}{2}-k_H\right)\right)=0$$
$$\Rightarrow x=\dfrac{2k_H-R}{(1-p)(2k_H+R)}$$
\end{enumerate}
\end{document}