\documentclass[a4paper]{article}
\usepackage{geometry}
\textwidth = 426pt
\usepackage{lmodern}
\usepackage{array}

\usepackage{multirow}
\usepackage{enumerate}
\usepackage{amssymb,amsmath}
\usepackage{ifxetex,ifluatex}
\usepackage{fixltx2e} % provides \textsubscript
\ifnum 0\ifxetex 1\fi\ifluatex 1\fi=0 % if pdftex
  \usepackage[T1]{fontenc}
  \usepackage[utf8]{inputenc}
\else % if luatex or xelatex
  \ifxetex
    \usepackage{mathspec}
  \else
    \usepackage{fontspec}
  \fi
  \defaultfontfeatures{Ligatures=TeX,Scale=MatchLowercase}
\fi
% use upquote if available, for straight quotes in verbatim environments
\IfFileExists{upquote.sty}{\usepackage{upquote}}{}
% use microtype if available
\IfFileExists{microtype.sty}{%
\usepackage[]{microtype}
\UseMicrotypeSet[protrusion]{basicmath} % disable protrusion for tt fonts
}{}
\PassOptionsToPackage{hyphens}{url} % url is loaded by hyperref
\usepackage[unicode=true]{hyperref}
\hypersetup{
            pdfborder={0 0 0},
            breaklinks=true}
\urlstyle{same}  % don't use monospace font for urls
\usepackage{longtable,booktabs}
% Fix footnotes in tables (requires footnote package)
\IfFileExists{footnote.sty}{\usepackage{footnote}\makesavenoteenv{long table}}{}
\IfFileExists{parskip.sty}{%
\usepackage{parskip}
}{% else
\setlength{\parindent}{0pt}
\setlength{\parskip}{6pt plus 2pt minus 1pt}
}
\setlength{\emergencystretch}{3em}  % prevent overfull lines
\providecommand{\tightlist}{%
  \setlength{\itemsep}{0pt}\setlength{\parskip}{0pt}}
\setcounter{secnumdepth}{0}
% Redefines (sub)paragraphs to behave more like sections
\ifx\paragraph\undefined\else
\let\oldparagraph\paragraph
\renewcommand{\paragraph}[1]{\oldparagraph{#1}\mbox{}}
\fi
\ifx\subparagraph\undefined\else
\let\oldsubparagraph\subparagraph
\renewcommand{\subparagraph}[1]{\oldsubparagraph{#1}\mbox{}}
\fi

% set default figure placement to htbp
\makeatletter
\def\fps@figure{htbp}
\makeatother

\author{Group 10: J. Bouten; T. Vissers; F. Strik; J. Bonthuis; M. Kroes}

\date{22 September, 2017}
\title{Game Theory 1: Problem Set 3}

\begin{document}

\maketitle
%\section{Problem Set 3}\label{problem-set-3}
\subsubsection{Problem 1}
\begin{enumerate}[(a)]
\item
Backward induction: \\ \\
Second stage (t=2)\\

$q_2=b(q_1,q_3)=\frac{1}{2}(a-c_2-q_1-q_3)$\\
$q_3=b(q_1,q_2)=\frac{1}{2}(a-c_3-q_1-q_2)$\\

$\hookrightarrow q_2=b(q_1)=\frac{1}{2}(a-c_2-q_1-(\frac{1}{2}(a-c_3-q_1-q_2)))$\\
$\frac{3}{4}q_2=\frac{1}{4}(a+c_3-q_1-2c_2)$\\

$q_2(q_1)=\begin{cases}
\frac{1}{3}(a+c_3-q_1-2c_2) & \quad \text{if }
q_1 \le a-2c_2+c_3\\
0 & \quad \text{if } q_1>a-2c_2+c_3
\end{cases}
$
\\ 

Similarly: $q_3(q_1)=\begin{cases}
\frac{1}{3}(a+c_2-q_1-2c_3) & \quad \text{if }
q_1 \le a-2c_3+c_2\\
0 & \quad \text{if } q_1>a-2c_3+c_2
\end{cases}
$\\ \\
First stage (t=1)\\
$\pi_1(q_1)=(a-c_1-q_2(q_1)-q_3(q_1))q_1$\\
$\pi_1(q_1)=(a-c_1-\frac{1}{3}(a+c_3-q_1-2c_2)-\frac{1}{3}(a+c_2-q_1-2c_3))q_1$\\
$\pi_1(q_1)=\frac{1}{3}(a-q_1+c_3+c_2-3c_1)q_1$\\
$FOC: \frac{\partial \pi_1}{\partial q_1}=\frac{1}{3}(a-2q_1+c_3+c_2-3c_1)=0$\\
$\hookrightarrow q_1^*=\frac{1}{2}(a-3c_1+c_2+c_3)$\\
This function along with the reaction functions of $q_2$ and $q_3$ as specified above is the SPNE of this Stackelberg game. \\ \\
$q_2^*=\frac{1}{3}(a+c_3-(\frac{1}{2}(a-3c_1+c_2+c_3))-2c_2)=\frac{1}{6}(a+c_3-5c_2+3c_1)$\\
Similarly: $q_3^*=\frac{1}{6}(a+c_2-5c_3+3c_1)$\\
\\Thus the outcome of the SPNE is:\\ $(q_1,q_2,q_3)=(\frac{1}{2}(a-3c_1+c_2+c_3),\frac{1}{6}(a+c_3-5c_2+3c_1),\frac{1}{6}(a+c_2-5c_3+3c_1))$ \clearpage 
\item
Backwards induction:\\ \\
Second stage (t=2)\\
$q_3=b(q_1,q_2)=
\begin{cases}
\frac{1}{2}(a-c_3-q_1-q_2) & \quad \text{if }
q_1+q_2 \le a-c_3\\
0 & \quad \text{if } q_1+q_2>a-c_3
\end{cases}$\\ \\
First stage (t=1)\\
$\pi_1=(a-c_1-q_1-q_2-\frac{1}{2}(a-c_3-q_1-q_2))q_1=\frac{1}{2}q_1(a-2c_1+c_3-q_1-q_2)$\\
$FOC: \frac{\partial\pi_1}{\partial q_1}=\frac{1}{2}(a-2c_1+c_3-2q_1-q_2)=0$\\
$q_1(q_2)=\frac{1}{2}(a-2c_1+c_3-q_2)$\\
Similarly, $q_2(q_1)=\frac{1}{2}(a-2c_2+c_3-q_1)$\\
\\
Substituting for $q_2$ in $q_1(q_2)$ we find:\\
$q_1=\frac{1}{2}(a-2c_1+c_3-(\frac{1}{2}(a-2c_2+c_3-q_1)))$\\
$q_1=\frac{1}{4}(a-4c_1+c_3+2c_2+q_1)$\\
$\frac{3}{4}q_1=\frac{1}{4}(a-4c_1+c_3+2c_2)$\\
$q_1^*=\frac{1}{3}(a-4c_1+c_3+2c_2)$\\
Similarly: $q_2^*=\frac{1}{3}(a-4c_2+c_3+2c_1)$\\
These functions along with the reaction function of firm 3 as specified above is the SPNE of this Stackelberg game\\
\\
$q_3^*=\frac{1}{2}(a-c_3-\frac{1}{3}(a-4c_1+c_3+2c_2)-\frac{1}{3}(a-4c-2+c_3+c_1))$\\
$q_3^*=\frac{1}{6}(a-5c_3+2c_1+2c_2)$\\
\\
Thus the outcome of the SPNE is:\\
$(q_1,q_2,q_3)=(\frac{1}{3}(a-4c_1+c_3+2c_2),\frac{1}{3}(a-4c_2+c_3+2c_1),\frac{1}{6}(a-5c_3+2c_1+2c_2))$\\
\\
\item

At time \(t = 1\), firm 1 acts as the Stackelberg leader in this market
and therefore chooses her quantity \(q_{1}\) first, based on the
reaction functions of firm 2 and 3. To maximize profits of firm 1, we
first have to find the reaction functions of both firms 2 and 3. The
reaction function of firm 3 is the same as we found before and thus
equals:

\[q_{3} = \frac{1}{2}(a - c_{3} - q_{1} - q_{2})\]

We can substitute this reaction function into the profit function for
firm 2 in order to find the reaction function of firm 2. Than we find:

\[\pi_{2} = \left( a - c_{2} - q_{1} - q_{2} - \frac{1}{2}\left( a - c_{3} - q_{1} - q_{2} \right) \right) \cdot q_{2}\]

\[FOC: \frac{\partial \pi_2}{\partial q_2}=\frac{1}{2}a - c_{2} + \frac{1}{2}c_{3} - \frac{1}{2}q_{1} - q_{2} = 0 \\
\  \]

\[q_{2} = \frac{1}{2}(a - 2c_{2} + c_{3} - q_{1})\]

This strategy set for player 2 can be filled into the best response
function for firm 3, this gives the new reaction function of firm 3:

\[q_{3} = \frac{1}{4}(a - 3c_{3} + 2c_{2} - q_{1})\]

We fill in \(q_{3}\) into the profit function of firm 1 to find the
first-stage decision of firm 1. This is as follows:

\[\pi_{1} = \left( a - c_{1} - q_{1} - \frac{1}{2}\left( a - 2c_{2} + c_{3} - q_{1} \right) - \frac{1}{4}(a - 3c_{3} - q_{1} + 2c_{2}) \right) \cdot q_{1}\]

\[\pi_{1} = \left( \frac{1}{4}a - c_{1} + \frac{1}{2}c_{2} + \frac{1}{4}c_{3} - \frac{1}{4}q_{1} \right) \cdot q_{1}\]

\[FOC: \frac{\partial \pi_1}{\partial q_1}= \frac{1}{4}a - c_{1} + \frac{1}{2}c_{2} + \frac{1}{4}c_{3} - \frac{1}{2}q_{1} = 0\]

\[q_{1}^{*} = \frac{1}{2}(a - 4c_{1} + 2c_{2} + c_{3})\ \]

This function along with the reactions function of firm 2 and 3 as
specified above, gives us the SPNE.

Further we obtain the second-stage decision when filling in
\(q_{1}^{*}\) into the reaction function of firm 2:

\[q_{2} = \frac{1}{2}\left( a - 2c_{2} + c_{3} - \frac{1}{2}\left( a - 4c_{1} + 2c_{2} + c_{3} \right) \right)\]

\[q_{2}^{*} = \frac{1}{4}\left( a - 6c_{2} + 4c_{1} + c_{3} \right)\]

Ultimately, we solve the third-stage decision substituting \(q_{1}^{*}\)
into the reaction function of firm 3:

\[q_{3} = \frac{1}{4}\left( a - 3c_{3} + 2c_{2} - \frac{1}{2}\left( a - 4c_{1} + 2c_{2} + c_{3} \right) \right)\]

\[q_{3}^{*} = \frac{1}{8}(a - 7c_{3} + 4c_{1} + 2c_{2})\]

Thus, the outcome along the SPNE path is therefore as follows:

\[\left( q_{1},\ q_{2},\ q_{3} \right) = \left( \frac{1}{2}\left( a - 4c_{1} + 2c_{2} + c_{3} \right),\frac{1}{4}\left( a - 6c_{2} + 4c_{1} + c_{3} \right),\ \frac{1}{8}\left( a - 7c_{3} + 4c_{1} + 2c_{2} \right) \right)\]

\item
$c_1=c_2=c_3=c \text{ and } a>c$\\
\\
$SPNE_{(a)}=(\frac{1}{2}(a-c),\frac{1}{6}(a-c),\frac{1}{6}(a-c))\rightarrow Q=\frac{5}{6}(a-c)$\\
$SPNE_{(b)}=(\frac{1}{3}(a-c),\frac{1}{3}(a-c),\frac{1}{6}(a-c))\rightarrow Q=\frac{5}{6}(a-c)$\\
$SPNE_{(c)}=(\frac{1}{2}(a-c),\frac{1}{2}(a-c),\frac{1}{8}(a-c))\rightarrow Q=\frac{9}{8}(a-c)$\\
\\
With symmetric cost levels and $a>c$ a welfare maximizing regulator would choose for the market structure defined in question (c). Total output is the greatest under this structure, therefore this is the closest to the perfectly competititve market (maximum total welfare) of all three structures (a), (b) and (c). 

$TW_{(c)}>TW_{(a)}=TW_{(b)}$ as: $\frac{9}{8}(a-c)>\frac{5}{6}(a-c)=\frac{5}{6}(a-c)$
\end{enumerate}
\subsubsection{Problem 2}
Firstly, we will attempt to find all the NE through backwards induction.

\begin{tabular}{ c c c c  } 

&& \multicolumn{2}{c}{Player 2} \\

&& A & B \\ 
\multirow{2}{4em}{Player 1} & S & \underline{6} , \underline{1} & 0 , 0  \\ 
 & T & 0 , 0 & \underline{4} , \underline{4} \\

\end{tabular}

The post LR decision subgame gives the following pure strategy NE: 
\begin{itemize}
\item
$(A,S)$ 
\item
$(B,T)$

Mixed strategy NE in the subgame:\\
\\
$u_1(S,\sigma_2)=6q$\\
$u_1(T,\sigma_2)=4-4q$\\
$6q=4-4q \text{ iff } q=\frac{2}{5}$\\
\\
$u_2(\sigma_1,A)=p$\\
$u_2(\sigma_1,B)=4-4p$\\
$p=4-4p \text{ iff }p=\frac{4}{5}$\\
\\
Thus, the mixed strategy NE in the post LR decision subgame is:
\item
$(\sigma_1,\sigma_2)=((\frac{4}{5},\frac{1}{5}),(\frac{2}{5},\frac{3}{5}))$ with $u_1(\sigma_1,\sigma_2)=\frac{12}{5}$\\
\end{itemize}
\begin{center}
\begin{tabular}{  p{18em}  p{18em}  } 
$x=1$ & $x=5$\\
\hline
1. \((S,A)\) gives player 1 a payoff of 6. So, player 1 will get 6 if
player 1 chooses L and 1 if player 1 chooses R. Player 1 will choose L
as this gives a higher payoff \((6 > 1)\).

SPNE: \((L,\ A,\ S)\)&
1.Player 1 will choose L as this gives a higher payoff \((6 > 5)\).

SPNE: \(\left( L ,A ,S \right)\)\\ \\ \\
\hline
2.\((T,B)\) gives player 1 a payoff of 4. So, player 1 will get 4 if
player 1 chooses L and 1 if player 1 chooses R. Player 1 will choose L
as this gives a higher payoff \((4 > 1)\).\\SPNE: \((L,B,T)\) & 
2.Player 1 will choose R as this gives a higher payoff \((5 > 4)\). 
SPNE: \((R,B,\ T)\)\\
\hline
3.\(\left( \left( \frac{4}{5},\frac{1}{5} \right),\left( \frac{4}{10},\frac{6}{10} \right) \right)\)
gives player 1 a payoff of \(\frac{12}{5}\). So, player 1 will get
\(\frac{12}{5}\) if player 1 chooses L and 1 if player 1 chooses R.
Player 1 will choose L as this gives a higher payoff
\((\frac{12}{5} > 1)\).
\\
SPNE:
\((L,\left( \frac{4}{5},\frac{1}{5} \right),(\frac{2}{5},\frac{3}{5}))\)&

3.Player 1 will choose R as this gives a higher payoff \((5 > 2.4)\).

SPNE:
\((R,\left( \frac{4}{5},\frac{1}{5} \right),\left( \frac{2}{5},\frac{3}{5} \right))\)\\
\end{tabular}
\end{center}








\end{document}