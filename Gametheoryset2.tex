\documentclass[a4paper]{article}
\usepackage{geometry}
\textwidth = 426pt
\usepackage{lmodern}
\usepackage{enumerate}
\usepackage{amssymb,amsmath}
\usepackage{ifxetex,ifluatex}
\usepackage{fixltx2e} % provides \textsubscript
\ifnum 0\ifxetex 1\fi\ifluatex 1\fi=0 % if pdftex
  \usepackage[T1]{fontenc}
  \usepackage[utf8]{inputenc}
\else % if luatex or xelatex
  \ifxetex
    \usepackage{mathspec}
  \else
    \usepackage{fontspec}
  \fi
  \defaultfontfeatures{Ligatures=TeX,Scale=MatchLowercase}
\fi
% use upquote if available, for straight quotes in verbatim environments
\IfFileExists{upquote.sty}{\usepackage{upquote}}{}
% use microtype if available
\IfFileExists{microtype.sty}{%
\usepackage[]{microtype}
\UseMicrotypeSet[protrusion]{basicmath} % disable protrusion for tt fonts
}{}
\PassOptionsToPackage{hyphens}{url} % url is loaded by hyperref
\usepackage[unicode=true]{hyperref}
\hypersetup{
            pdfborder={0 0 0},
            breaklinks=true}
\urlstyle{same}  % don't use monospace font for urls
\usepackage{longtable,booktabs}
% Fix footnotes in tables (requires footnote package)
\IfFileExists{footnote.sty}{\usepackage{footnote}\makesavenoteenv{long table}}{}
\IfFileExists{parskip.sty}{%
\usepackage{parskip}
}{% else
\setlength{\parindent}{0pt}
\setlength{\parskip}{6pt plus 2pt minus 1pt}
}
\setlength{\emergencystretch}{3em}  % prevent overfull lines
\providecommand{\tightlist}{%
  \setlength{\itemsep}{0pt}\setlength{\parskip}{0pt}}
\setcounter{secnumdepth}{0}
% Redefines (sub)paragraphs to behave more like sections
\ifx\paragraph\undefined\else
\let\oldparagraph\paragraph
\renewcommand{\paragraph}[1]{\oldparagraph{#1}\mbox{}}
\fi
\ifx\subparagraph\undefined\else
\let\oldsubparagraph\subparagraph
\renewcommand{\subparagraph}[1]{\oldsubparagraph{#1}\mbox{}}
\fi

% set default figure placement to htbp
\makeatletter
\def\fps@figure{htbp}
\makeatother

\author{Group 10: J. Bouten; T. Vissers; F. Strik; J. Bonthuis; M. Kroes}

\date{22 September, 2017}
\title{Game Theory 1: Problem Set 2}

\begin{document}

\maketitle
%\section{Problem Set 2}\label{problem-set-2}

\subsubsection{Problem 1}\label{problem1}

Homogeneous cost levels:\\
$\pi_i=(100-q_i-q_{-i}-20)q_i=80q_i-q_iq_{-i}-q_i^2$\\ 
FOC: $\frac{\partial\pi_i}{\partial q_i}=80q_i-q_{-i}-2q_i=0 \rightarrow q_i=40-\frac{1}{2}q_{-i}$\\
Symmetry: $q_i=q_j \rightarrow q_i=40-\frac{1}{2}(q_i)  \rightarrow q_i=\frac{80}{3}$
\\ 
Thus the Nash equilibrium in the homogeneous cost model is $(q_1,q_2)=(\frac{80}{3},\frac{80}{3})$
\\ $HHI_{old}=\left(\frac{1}{2}\right)^2\cdot2=\frac{1}{2}$\\ \\
Heterogeneous cost levels: \\ $\pi_1=70q_1-q_1q_2-q_1^2$\\
FOC: $\frac{\partial\pi_1}{\partial q_1}=70q_1-q_2-2q_1=0 \rightarrow q_1=35-\frac{1}{2}q_2 $\\
$\pi_2=90q_2-q_1q_2-q_2^2$\\
FOC: $\frac{\partial\pi_2}{\partial q_2}=90-q_1-2q_2=0 \rightarrow q_2=45-\frac{1}{2}q_1$\\
$q_1=35-\frac{1}{2}(45-\frac{1}{2}q_1) \rightarrow q_1=\frac{50}{3}$\\
$q_2=45-\frac{1}{2}\cdot\frac{50}{3} \rightarrow q_2=\frac{110}{3}$\\
Thus the Nash equilibrium in the heterogeneous cost model is $(q_1,q_2)=(\frac{50}{3},\frac{110}{3})$\\
$HHI_{new}=\left(\frac{5}{16}\right)^2+\left(\frac{11}{16}\right)^2=\frac{73}{128}>\frac{1}{2}$

The Herfidahl-Hirschmann Index is now larger than it was before, the market concentration has become larger. 

\subsubsection{Problem 2}\label{problem2}
\begin{enumerate}[(a)]
\item
\underline{Firm 1:} \\$\pi_1=(80-q_1-q_2-q_3-q_4)q_1$\\
$FOC : 80-2q_1-q_2-q_3-q_4=0$\\
$BR_1: q_1(q_2,q_3,q_4)=40-\frac{1}{2}(q_2+q_3+q_4)$\\
Similarly:\\
\underline{Firm 2:}
$BR_2: q_2(q_1,q_3,q_4)=40-\frac{1}{2}(q_1+q_3+q_4)$\\
\underline{Firm 3:}
$BR_3: q_3(q_1,q_2,q_4)=40-\frac{1}{2}(q_1+q_2+q_4)$\\
\\
\underline{Firm 4:}
$\pi_4=(80-\gamma-q_1-q_2-q_3-q_4)q_4$\\
$FOC: 80-\gamma-q_1-q_2-q_3-2q_4=0$\\
$BR_4: q_4(q_1,q_2,q_3,\gamma)=40-\frac{1}{2}(\gamma+q_1+q_2+q_3)$\\
\\
Due to symmetry: $q_1^*=q_2^*=q_3^*=q_j$\\
$q_j=40-q_j-\frac{1}{2}q_4 \rightarrow q_j(q_4)=20-\frac{1}{4}q_4$\\
$q_4=40-\frac{1}{2}\gamma-\frac{3}{2}(20-\frac{1}{4}q_4) \rightarrow q_4^*(\gamma)=16-\frac{4}{5}\gamma$
\\
\\
$q_1^*(\gamma)=q_2^*(\gamma)=q_3^*(\gamma)=16+\frac{1}{5}\gamma$ \& $q_4^*(\gamma)=16-\frac{4}{5}\gamma\rightarrow Q=50+\frac{1}{10}\gamma$\\

NE:\\ $[q_1=q_2=q_3,q_4]=[(16+\frac{1}{5}\gamma), (16-\frac{4}{5}\gamma)]$\\

$Q=q_1+q_2+q_3+q_4=64-\frac{1}{5}\gamma$\\

$\pi_1=(16+\frac{1}{5}\gamma)^2$\\
$\pi_4=(16-\frac{4}{5}\gamma)^2$\\ \\
$[\pi_1=\pi_2=\pi_3,\pi_4]=[((16+\frac{1}{5}\gamma)^2),((16-\frac{4}{5}\gamma)^2)]$\\
\\
To find the constraints, we solve $P=100-64+\frac{1}{5}\gamma\ge Max[20,20+\gamma]$, meaning that the price must be greater than the variable cost of all firms given the equilibrium values for $Q = q_1^*+q_2^*+q_3^*+q_4^*$.\\
Constraints that follow from setting prices equal to variable cost levels for all firms are: \\
$\gamma\ge-80$ \& $\gamma\le20 \rightarrow$ the constraints for $\gamma$ are: $-20<\gamma\le20$. 
\item
\underline{Firm 1\&2:} The merged entity\\
$\pi_{12}=(80-q_{12}-q_3-q_4)q_{12}$\\
$FOC: 80-2q_{12}-q_3-q_4=0$\\
$BR_{12}:q_{12}(q_3,q_4)=40-\frac{1}{2}(q_3+q_4)$\\
Similarly:\\
\underline{Firm 3:} $BR_3:q_3(q_{12},q_4)=40-\frac{1}{2}(q_{12}+q_4)$\\ \\
\underline{Firm 4:} $\pi=(80-\gamma-q_{12}-q_3-q_4)q_4$\\
$FOC:80-\gamma-q_{12}-q_3-2q_4=0$\\
$BR_4:q_4(q_{12},q_3,\gamma)=40-\frac{1}{2}(\gamma+q_{12}+q_3)$\\
\\
Due to symmetry: $q_{12}^*=q_3^*=q_j$\\
$q_j=40-\frac{1}{2}(q_j+q_4)\rightarrow q_j(q_4)=\frac{80}{3}-\frac{1}{3}q_4$\\
$q_4=40-\frac{1}{2}\gamma-\frac{80}{3}+\frac{1}{3}q_4 \rightarrow q_4^*(\gamma)=20-\frac{3}{4}\gamma$\\
$q_{12}^*(\gamma)=q_3^*(\gamma)=\frac{80}{3}-\frac{1}{3}(20-\frac{3}{4}\gamma)=20+\frac{1}{4}\gamma$\\
\\
NE:\\
$[q_{12}=q_3,q_4]=[20+\frac{1}{4}\gamma , 20-\frac{3}{4}\gamma]$\\
\\
$Q=q_{12}+q_3+q_4=60-\frac{1}{4}\gamma$\\
\\Profits are calculated as $(P-MC)q_{12}$:\\
$\pi_{12}=((100-(60-\frac{1}{4}\gamma))-20)(20+\frac{1}{4}\gamma)=(20+\frac{1}{4}\gamma)^2$\\
This profit level is always greater than the profit level for firm 1 or 2 in the individual case. However, it is always smaller than the sum of the profit levels for firm 1 and 2 in the pre-merger case. Therefore, the merger is not profitable as joint profits fall after merging. \\
$2\cdot(16+\frac{1}{5}\gamma)^2> (20+\frac{1}{4}\gamma) $ if $ \gamma\ne -80$  \space  ($\gamma$ will never equal $-80$ as $\gamma>-20$)
\item
We assume that the firm takes the lowest cost level. If $\gamma$ is positive this would mean the new cost would be $C(q_{14})=20$.\\
\underline{Firm 1\&4:} The merged entity\\
$\pi_{14}=(80-q_{14}-q_2-q_3)q_{14}$\\
$FOC:80-2q_{14}-q_2-q_3=0$\\
$BR_{14}:q_{14}(q_2,q_3)=40-\frac{1}{2}(q_2+q_3)$\\
Similarly:\\
\underline{Firm 2:} $BR_2:q_2(q_{14},q_3)=40-\frac{1}{2}(q_{14}+q_3)$\\
And:
\\
\underline{Firm 3:} $BR_3:q_3(q_{14},q_2)=40-\frac{1}{2}(q_{14}+q_2)$\\
\\
Due to symmetry: $q_{14}=q_2=q_3=q$\\
Now: $q=40-q \rightarrow q=20$\\
\\
NE: $[q_{14},q_2,q_3]=[20,20,20]$\\
\\
$Q=3q=60$\\
\\
$\pi_{14}=\pi_2=\pi3=20^2=400$\\
The profit level for the merged firm is now 400. Under the pre-merger conditions, the joint profit was equal to
 $(16+\frac{1}{5}\gamma)^2+(16-\frac{4}{5}\gamma)^2$. We can compare these profit levels:\\
Intersections:
$(16+\frac{1}{5}\gamma)^2+(16-\frac{4}{5}\gamma)^2=512+\frac{32-128}{5}\gamma+\frac{17}{25}\gamma^2=400$
\\$17\gamma^2-480\gamma+2800=0$\\
$D=b^2-4ac=480^2-4\cdot17\cdot2800=40000\rightarrow\sqrt{D}=200$\\
$\gamma=\frac{480\pm200}{34}$\\
$ \gamma=20 $ or $ \gamma=\frac{140}{17}$
\\ \\
After checking with a sign scheme we find that the merger can be profitable, this is the case if $\frac{140}{17}<\gamma<20$
\\
For firm 2 the profit changes from $\pi_2=(16+\frac{1}{5}\gamma)^2$ before the merger to $\pi_2=400$ after the merger. The conditions on $\gamma$ for the merger to be profitable for firm 2 can be calculated as follows:\\
\\
$16+\frac{1}{5}\gamma=20 \rightarrow \gamma=20$\\
The constraint that follows is $\gamma<20$. \\The conditions on $\gamma$ are $\gamma>-20$ and $\gamma<20$ for the profits of firm 2 to rise as a result of the merger.  

\end{enumerate}

\subsubsection{Problem 3}\label{problem3}
\begin{enumerate}[(a)]
\item
$\pi_i=p_iq_i-F=q_i-q_i^2-\theta q_i \sum\limits_{j\ne i} q_j-F$\\
FOC:$ \frac{\partial\pi_i}{\partial q_i}=1-2q_i-\theta\sum\limits_{j\ne i}q_j=0 \rightarrow q_i=\frac{1}{2}-\frac{\theta}{2}\sum\limits_{j \ne i}q_j$\\
Symmetry: $q_i=q_j \rightarrow q_i=\frac{1}{2}-\frac{\theta}{2}(n-1)q_i \rightarrow q_i=\frac{1}{2+\theta(n-1)}$
\\Symmetric Nash equilibrium: $(q_i^*)=(\frac{1}{2+\theta(n-1)})$
\item
Firms will enter as long as they can generate positive profits by entering the market. This then determines the number of firms that will enter.\\
We find the equilibrium number of firms ($n^*$) by setting profits ($\pi_i(q_i,q_j,n,\theta)=\pi_i(q_i^*,q_i^*,n,\theta)$) equal to $0$ and then solving for $n$.\\
$\pi_i=\left(\frac{1}{2+\theta(n-1)}\right)\left(1-(\theta(n-1)+1)\left(\frac{1}{2+\theta(n-1)}\right)\right)-F=0$\\
$\left(\frac{1}{2+\theta(n-1)}\right)\left(\frac{2+\theta(n-1)}{2+\theta(n-1)}-\frac{1+\theta(n-1)}{2+\theta(n-1)}\right)=F$\\$F=\left(\frac{1}{2+\theta(n-1)}\right)^2$\\
$\frac{1}{\sqrt{F}}-2=\theta(n-1)\rightarrow n^*=\frac{1-2\sqrt{F}}{\sqrt{F}\theta}+1$\\ 
\\
From this relationship we can see that the equilibrium number of firms that enter the market is determined by the fixed cost level as well as the parameter $\theta$. The parameter $\theta$ negatively influences the equilibrium number of firms. The rationale behind this is that if $\theta$ would be equal to 0, each firm would be a monopolist with its own market, without influence from other firms. This causes the $n^*$ to go to infinity if $\theta$ goes to 0, as there would always be room for other firms to enjoy positive profits if $\theta$ would equal 0. As $\theta$ rises, the market becomes more competitive and thus it becomes harder to reach positive profit levels when the influence of other firms becomes greater.  




\end{enumerate}

\subsubsection{Problem 4}\label{problem4}

First, by inspection we can clearly see there are two pure strategy Nash
Equilibria: {[}B,L{]} and {[}T,R{]}

Now we will attempt to find mixed-strategy Nash equilibria by using the
CMS-NE approach.

First of all, we'll consider the possibility of an equilibrium in which
Player 1's strategy is pure, whereas Player 2's strategy assigns
positive probabilities to two or more strategies.

So if P1 is to play the pure strategy T, we can find a mixed-strategy
Nash equilibrium, since P2 is indifferent in playing M and R. According
to the CMS-NE, P2 should play M and R both with probability 0.5 to
satisfy the propositions. So the first mixed-strategy Nash equilibrium
is as follows: {[}(1,0),(0,0.5,0.5){]}.

Also if P2 is to play the pure strategy B, we can find a mixed-strategy
Nash equilibrium, since P2 is indifferent in playing L and R. According
to the CMS-NE, P2 should play L and R both with probability 0.5 to
satisfy the propositions. So the second mixed-strategy Nash equilibrium
is as follows: {[}(0,1),(0.5,0,0.5){]}.

Next, we consider the probability of an equilibrium in which P1's
strategy assigns positive probabilities to both actions and P2's
strategy assigns positive probability to 2 of her 3 actions. This gives
\(\sigma_{1} = (p,\ 1 - p)\) with three pairs of possibilities for P2's
actions for which we check if the CMS-NE propositions hold.

1. P2 plays L and M

\[u_{2}\left( \sigma_{1},\ L \right) = u_{2}\left( \sigma_{1},\ M \right) \geq u_{2}(\sigma_{1},\ R)\]

\[2p + 2\left( 1 - p \right) = 3p + \left( 1 - p \right) \geq 3p + 2\left( 1 - p \right)\]

For proposition 1 to hold it follows that \(p = 0.5\). This, however,
does not satisfy proposition 2, since
\(u_{2}\left( \sigma_{1},R \right) > u_{2}(\sigma_{1},M)\). Thus, there
is no equilibrium of this type.

\[u_{2}\left( \sigma_{1},\ M \right) = u_{2}\left( \sigma_{1},\ R \right) \geq u_{2}(\sigma_{1},\ L)\]

\[3p + \left( 1 - p \right) = 3p + 2\left( 1 - p \right) \geq 2p + 2\left( 1 - p \right)\]

For proposition 1 to hold it follows that \(p = 1\). Proposition 2 holds
as well, so P2 plays \(\sigma_{2} = (0,q,1 - q)\). We solve this for
\emph{q}, giving:

\[3q + 2\left( 1 - q \right) = 2q + 2\left( 1 - q \right)\]

\[q = 0.5\]

So, since P1 plays T with probability \(p = 1\), we obtain a similar
result as the first mixed-strategy Nash equilibrium we found:
{[}(1,0),(0,0.5,0.5){]}

\[u_{2}\left( \sigma_{1},\ R \right) = u_{2}\left( \sigma_{1},\ L \right) \geq u_{2}(\sigma_{1},\ M)\]

\[3p + 2\left( 1 - p \right) = 2p + 2\left( 1 - p \right) \geq 3p + (1 - p)\]

For proposition 1 to hold it follows that \(p = 0\). Proposition 2 holds
as well. So P2 plays \(\sigma_{2} = (q,0,1 - q)\). We solve this for 1,
giving:

\[2q + 2\left( 1 - q \right) = 3q + \left( 1 - q \right)\]

\[q = 0.5\]

So, since P2 plays B with probability \(p = 1,\ \)we obtain a similar
result as the second mixed-strategy Nash equilibrium we found:
{[}(0,1),(0.5,0,0.5){]}.

Lastly, there is the possibility that P1 assigns positive probabilities
to both her actions and P2 to all three of her actions such that:

\[u_{2}\left( \sigma_{1},L \right) = u_{2}\left( \sigma_{1},M \right) = u_{2}(\sigma_{1},R)\]

Then the following has to hold:

\[2p + 2\left( 1 - p \right) = 3p + \left( 1 - p \right) = 3p + 2(1 - p)\]

There is, however, no value of \(\text{p\ }\)for which P2's expected
payoffs to her three actions are equal. So there does not exist a Nash
equilibrium here.

Taken together, we conclude that the game has four mixed-strategy Nash
equilibria:

{[}(1,0),(0,0,1){]}, {[}(0,1),(1,0,0){]}, {[}(1,0),(0,0.5,0.5){]} and
{[}(0,1),(0.5,0,0.5){]}.
\end{document}