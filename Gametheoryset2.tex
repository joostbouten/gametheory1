\documentclass[a4paper]{article}
\usepackage{geometry}
\textwidth = 426pt
\usepackage{lmodern}
\usepackage{enumerate}
\usepackage{amssymb,amsmath}
\usepackage{ifxetex,ifluatex}
\usepackage{fixltx2e} % provides \textsubscript
\ifnum 0\ifxetex 1\fi\ifluatex 1\fi=0 % if pdftex
  \usepackage[T1]{fontenc}
  \usepackage[utf8]{inputenc}
\else % if luatex or xelatex
  \ifxetex
    \usepackage{mathspec}
  \else
    \usepackage{fontspec}
  \fi
  \defaultfontfeatures{Ligatures=TeX,Scale=MatchLowercase}
\fi
% use upquote if available, for straight quotes in verbatim environments
\IfFileExists{upquote.sty}{\usepackage{upquote}}{}
% use microtype if available
\IfFileExists{microtype.sty}{%
\usepackage[]{microtype}
\UseMicrotypeSet[protrusion]{basicmath} % disable protrusion for tt fonts
}{}
\PassOptionsToPackage{hyphens}{url} % url is loaded by hyperref
\usepackage[unicode=true]{hyperref}
\hypersetup{
            pdfborder={0 0 0},
            breaklinks=true}
\urlstyle{same}  % don't use monospace font for urls
\usepackage{longtable,booktabs}
% Fix footnotes in tables (requires footnote package)
\IfFileExists{footnote.sty}{\usepackage{footnote}\makesavenoteenv{long table}}{}
\IfFileExists{parskip.sty}{%
\usepackage{parskip}
}{% else
\setlength{\parindent}{0pt}
\setlength{\parskip}{6pt plus 2pt minus 1pt}
}
\setlength{\emergencystretch}{3em}  % prevent overfull lines
\providecommand{\tightlist}{%
  \setlength{\itemsep}{0pt}\setlength{\parskip}{0pt}}
\setcounter{secnumdepth}{0}
% Redefines (sub)paragraphs to behave more like sections
\ifx\paragraph\undefined\else
\let\oldparagraph\paragraph
\renewcommand{\paragraph}[1]{\oldparagraph{#1}\mbox{}}
\fi
\ifx\subparagraph\undefined\else
\let\oldsubparagraph\subparagraph
\renewcommand{\subparagraph}[1]{\oldsubparagraph{#1}\mbox{}}
\fi

% set default figure placement to htbp
\makeatletter
\def\fps@figure{htbp}
\makeatother

\author{Group 10: J. Bouten; T. Vissers; F. Strik; J. Bonthuis; M. Kroes}

\date{22 September, 2017}
\title{Game Theory 1: Problem Set 2}

\begin{document}

\maketitle
%\section{Problem Set 2}\label{problem-set-2}

\subsubsection{Problem 1}\label{problem1}

Homogeneous cost levels:\\
$\pi_i=(100-q_i-q_{-i}-20)q_i=80q_i-q_iq_{-i}-q_i^2$\\ 
FOC: $\frac{\partial\pi_i}{\partial q_i}=80q_i-q_{-i}-2q_i=0 \rightarrow q_i=40-\frac{1}{2}q_{-i}$\\
Symmetry: $q_i=q_j \rightarrow q_i=40-\frac{1}{2}(q_i)  \rightarrow q_i=\frac{80}{3}$
\\ 
Thus the Nash equilibrium in the homogeneous cost model is $(q_1,q_2)=(\frac{80}{3},\frac{80}{3})$
\\ $HHI_{old}=\left(\frac{1}{2}\right)^2\cdot2=\frac{1}{2}$\\ \\
Heterogeneous cost levels: \\ $\pi_1=70q_1-q_1q_2-q_1^2$\\
FOC: $\frac{\partial\pi_1}{\partial q_1}=70q_1-q_2-2q_1=0 \rightarrow q_1=35-\frac{1}{2}q_2 $\\
$\pi_2=90q_2-q_1q_2-q_2^2$\\
FOC: $\frac{\partial\pi_2}{\partial q_2}=90-q_1-2q_2=0 \rightarrow q_2=45-\frac{1}{2}q_1$\\
$q_1=35-\frac{1}{2}(45-\frac{1}{2}q_1) \rightarrow q_1=\frac{50}{3}$\\
$q_2=45-\frac{1}{2}\cdot\frac{50}{3} \rightarrow q_2=\frac{110}{3}$\\
Thus the Nash equilibrium in the heterogeneous cost model is $(q_1,q_2)=(\frac{50}{3},\frac{110}{3})$\\
$HHI_{new}=\left(\frac{5}{16}\right)^2+\left(\frac{11}{16}\right)^2=\frac{73}{128}>\frac{1}{2}$

The Herfidahl-Hirschmann Index is now larger than it was before, the market concentration has become larger. 

\subsubsection{Problem 2}\label{problem2}

\subsubsection{Problem 3}\label{problem3}
\begin{enumerate}[\textbf{(a)}]
\item
$\pi_i=p_iq_i-F=q_i-q_i^2-\theta q_i \sum\limits_{j\ne i} q_j-F$\\
FOC:$ \frac{\partial\pi_i}{\partial q_i}=1-2q_i-\theta\sum\limits_{j\ne i}q_j=0 \rightarrow q_i=\frac{1}{2}-\frac{\theta}{2}\sum\limits_{j \ne i}q_j$\\
Symmetry: $q_i=q_j \rightarrow q_i=\frac{1}{2}-\frac{\theta}{2}(n-1)q_i \rightarrow q_i=\frac{1}{2+\theta(n-1)}$
\\Symmetric Nash equilibrium: $(q_i^*)=(\frac{1}{2+\theta(n-1)})$
\item
Firms will enter as long as they can generate positive profits by entering the market. This then determines the number of firms that will enter.\\
We find the equilibrium number of firms ($n^*$) by setting profits ($\pi_i(q_i,q_j,n,\theta)=\pi_i(q_i^*,q_i^*,n,\theta)$) equal to $0$ and then solving for $n$.\\
$\pi_i=\left(\frac{1}{2+\theta(n-1)}\right)\left(1-(\theta(n-1)+1)\left(\frac{1}{2+\theta(n-1)}\right)\right)-F=0$
$\rightarrow n=$

\end{enumerate}

\subsubsection{Problem 4}\label{problem4}


\end{document}