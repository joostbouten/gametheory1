\documentclass[]{article}
\usepackage{lmodern}
\usepackage{amssymb,amsmath}
\usepackage{ifxetex,ifluatex}
\usepackage{fixltx2e} % provides \textsubscript
\ifnum 0\ifxetex 1\fi\ifluatex 1\fi=0 % if pdftex
  \usepackage[T1]{fontenc}
  \usepackage[utf8]{inputenc}
\else % if luatex or xelatex
  \ifxetex
    \usepackage{mathspec}
  \else
    \usepackage{fontspec}
  \fi
  \defaultfontfeatures{Ligatures=TeX,Scale=MatchLowercase}
\fi
% use upquote if available, for straight quotes in verbatim environments
\IfFileExists{upquote.sty}{\usepackage{upquote}}{}
% use microtype if available
\IfFileExists{microtype.sty}{%
\usepackage[]{microtype}
\UseMicrotypeSet[protrusion]{basicmath} % disable protrusion for tt fonts
}{}
\PassOptionsToPackage{hyphens}{url} % url is loaded by hyperref
\usepackage[unicode=true]{hyperref}
\hypersetup{
            pdfborder={0 0 0},
            breaklinks=true}
\urlstyle{same}  % don't use monospace font for urls
\usepackage{longtable,booktabs}
% Fix footnotes in tables (requires footnote package)
\IfFileExists{footnote.sty}{\usepackage{footnote}\makesavenoteenv{long table}}{}
\IfFileExists{parskip.sty}{%
\usepackage{parskip}
}{% else
\setlength{\parindent}{0pt}
\setlength{\parskip}{6pt plus 2pt minus 1pt}
}
\setlength{\emergencystretch}{3em}  % prevent overfull lines
\providecommand{\tightlist}{%
  \setlength{\itemsep}{0pt}\setlength{\parskip}{0pt}}
\setcounter{secnumdepth}{0}
% Redefines (sub)paragraphs to behave more like sections
\ifx\paragraph\undefined\else
\let\oldparagraph\paragraph
\renewcommand{\paragraph}[1]{\oldparagraph{#1}\mbox{}}
\fi
\ifx\subparagraph\undefined\else
\let\oldsubparagraph\subparagraph
\renewcommand{\subparagraph}[1]{\oldsubparagraph{#1}\mbox{}}
\fi

% set default figure placement to htbp
\makeatletter
\def\fps@figure{htbp}
\makeatother

\author{Group 10: J. Bouten; T. Vissers; F. Strik; J. Bonthuis; M. Kroes}

\date{4 September, 2017}
\title{Game Theory 1: Problem Set 1}

\begin{document}

\maketitle
%\section{Problem Set 1}\label{problem-set-1}



\subsubsection{Problem 1}\label{problem-1}

Firstly we delete vectors \textbf{A} (dom. by \textbf{B} \& \textbf{C}), \textbf{a} (dom. by  \textbf{b}), \textbf{E} (dom. by\textbf{B}), \textbf{d} (dom. by \textbf{b})
and \textbf{D} (dom. by \textbf{B}). These vectors are all strictly dominated.

The following matrix now remains:

\begin{tabular}{| c | c |}
  \hline			
  88, 130 &  88, 130 \\
\hline
  58, 160 & 128, 90 \\
  \hline  
\end{tabular}

From this game we can see that \textbf{c} is weakly dominated by \textbf{b}, after
deleting \textbf{c} we find that \textbf{C} is strictly dominated \textbf{B}. The
remaining cell is \textbf{[B,b]}, based on deleting dominated
strategies the recommended strategy would be to play \textbf{B} for
player 1 and \textbf{b} for player 2.


\subsubsection{Problem 2}\label{problem-2}

If first weakly dominated strategy \textbf{T} is deleted, the outcome will be one in row \textbf{B} (meaning the outcome will be either cell \textbf{[B,L]} or cell \textbf{[B,R]}). \\

On the other hand, if weakly dominated strategy \textbf{L} is deleted first, the outcome will be one in column \textbf{R} (meaning the outcome will be either cell \textbf{[T,R]} or cell \textbf{[B,R]}).
The second mover has no incentive to choose as the outcome for the second mover will always equal 0. \\

In the case that weakly dominated strategies can be deleted simultaneously, the outcome will be \textbf{[B,R]}.
\subsubsection{Problem 3}\label{problem-3}

One solution is first deleting strictly dominated strategy \textbf{H} for
player one, which leaves the matrix

\begin{tabular}{| c | c |}
  \hline			
  -1, 1 &  1, -1 \\
\hline
  3, 1 & 2, 1 \\
  \hline  
\end{tabular}

In this matrix, we can now delete weakly dominated strategy \textbf{T} for
player two. Lastly we can delete strictly dominated strategy \textbf{T}
for player one, leaving the cell \textbf{[O,H]} as our first possible
solution.

Another solution is starting with deleting strictly dominated strategy
\textbf{T} for player one. Then deleting weakly dominated strategy
\textbf{H} for player two and lastly deleting strictly dominated strategy
\textbf{H} for player one, leaving the second possible solution \textbf{[O,T]}.

\subsubsection{Problem 4}\label{problem-4}

In column \textbf{L}, the highest outcome for player one is given by
playing strategy \textbf{T}. In column \textbf{M} the highest outcome for
player one is given by playing strategy \textbf{B}. In column \textbf{R} the
highest outcome for player one is given by playing strategy \textbf{M}.
From the viewpoint of player one we now have three candidates for Nash
equilibria: \textbf{[T,L]}, \textbf{[B,M]} and \textbf{[M,R]}.\strut

We now have to check whether player two would deviate from these cells,
for this we run the same procedure for player two. In row \textbf{T}, the
highest outcome for player two is given by playing strategy \textbf{L}. In
row \textbf{M} the highest outcome for player two is given by playing
strategy \textbf{M}. in row \textbf{B} the highest outcome for player two is
given by playing strategy \textbf{M}. From this information we can see
that \textbf{[M,R]} is not a Nash equilibrium, because player two would
deviate.\strut
There are two Nash equilibria: cell \textbf{[T,L]} and cell \textbf{[B,M]}.\strut

\subsubsection{Problem 5}\label{problem-5}

\paragraph{(a)}\label{a}

If \textbf{[A,a]} is played, player 2 has an incentive to change to
\textbf{[A,b]},\\
If \textbf{[A,b]} is played, player 1 has an incentive to change to
\textbf{[B,b]},\\
If \textbf{[B,B]} is played, player 2 has an incentive to change to
\textbf{[B,a]},\\
If \textbf{[B,a]} is played, player 1 has an incentive to change to
\textbf{[A,a]},\\
If \textbf{[A,a]} is played, \ldots{}\\
And so forth: the game is never in equilibrium as there is always a
possible deviation leading to a higher outcome to one of the two
players. The best responses of player 1 and player 2 never intersect, therefore there is no cell which is a Nash equilibrium in pure strategies.

\paragraph{(b)}\label{b}

There exists a Nash equilibrium in mixed strategies if the expected
payoff of each cell is equal. 

To check whether $[(\frac{1}{2},\frac{1}{2}),(\frac{1}{2},\frac{1}{2})] $ is a Nash equilibrium in mixed strategies we will solve for $\sigma$ in the following  equation. $\sigma$ denotes the percentage of times A is played. We have two utility functions for player 1 which we set equal to each other to find the equilibrium value of $\sigma$. The first utility function is the utility for player 1 if player 2 chooses to play \textbf{a}. The second utility function is the utility for player 1 if player 2 plays \textbf{b}.  \\

$
U_a=\sigma(1)+(1-\sigma)(-1)=U_b=\sigma(-1)+(1-\sigma)(1) $\\
\\
Solving for $\sigma$ we find:
$\sigma=\frac{1}{2}$

We can do the same for player 2, but since this game is the same for player 2 this will yield the same outcome. Thus leading to the Nash equilibrium indeed being $[(\frac{1}{2},\frac{1}{2}),(\frac{1}{2},\frac{1}{2})] $.


\subsubsection{Problem 6}\label{problem-6}

\paragraph{(a)}\label{a}

$P(Q)=a-b\cdot Q$ with $MC=0$

$Q=q_1+...+q_n$ 

Due to symmetry $q_i=q_1=q_2=...=q_n$

$\pi_i=(a-b\cdot Q)q_i-cq_i$

We set the derivative of this profit function equal to 0 to find the best response function of firm i.

$\frac{\partial\pi_i}{\partial q_i}=a-2q_i-bQ_{n-i}-c=0$

From this equation we can get the best response of firm i:

$q_i^*=\frac{a-c}{b(1+n)}$

This is a unique Nash equilibrium, deviation from $q_i^*$ of any firm can only lead to a negative effect on individual profits. Cournot equilibria are always unique due to convergence of $q_i$ towards the Cournot equilibrium from any starting point of $q_i$. It is a unique Nash equilibrium because it is the only point where the best response functions of all players intercept.

\paragraph{(b)}

As $n\rightarrow\infty$, \\
$q_i^*=\frac{a-c}{b(1+n)}\rightarrow 0$\\
$P=\frac{a+nc}{1+n}\rightarrow c$    (marginal costs)\\
$\pi_i=\frac{(a-c)^2}{b(1+n)^2}\rightarrow 0$ \\

Economic intuition:\\
An increase in the number of firms increases competition. $n\rightarrow\infty$ is the outcome of the perfectly competitive market with prices equal to marginal costs. No firm has the market power to obtain any profits. 
The outcomes under a cournot market with infinitely many firms are equal to the outcomes in a market with Bertrand competition, where firms compete in prices until the price equals their marginal cost level. (Assuming homogeneous products and cost levels) 
\end{document}
